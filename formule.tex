\documentclass[12pt, a4paper]{article}
\usepackage[utf8]{inputenc}
\usepackage[romanian]{babel}
\usepackage{amsmath}
\usepackage{amssymb}
\usepackage{amsfonts}
\usepackage{geometry}
\geometry{a4paper, margin=1in}

\title{Formular de Analiza a Datelor si ACP (cu Explicatii)}
\author{Extras din documente}
\date{\today}

\begin{document}

\maketitle

\section{Document 1: Analiza Preliminara a Datelor}

Varianta (caz discret): Media ponderata a patratelor abaterilor de la medie.
\begin{equation}
\sigma^2 = \sum_{x \in R} (x - \mu)^2 \cdot f(x)
\end{equation}

Varianta (pentru esantion / repartitie uniforma): Media patratelor abaterilor de la medie.
\begin{equation}
\sigma^2 = \frac{1}{n} \sum_{i=1}^{n} (x_i - \mu)^2
\end{equation}

Abaterea standard: Radacina patrata a variantei; masoara împrastierea în unitatile variabilei.
\begin{equation}
\sigma = \sqrt{\sigma^2}
\end{equation}

Coeficientul de variatie: Raportul dintre abaterea standard si medie; o masura relativa a împrastierii.
\begin{equation}
C_v = \frac{\sigma}{\mu}
\end{equation}

\subsection{Indicatori de Forma}

Asimetria (Skewness): Masoara gradul de simetrie a distributiei (bazata pe momentul 3).
\begin{equation}
S = \frac{MC_3}{\sigma^3}
\end{equation}

Aplatizarea (Kurtosis): Masoara 'boltirea' distributiei (bazata pe momentul 4).
\begin{equation}
K = \frac{MC_4}{\sigma^4} \quad \text{sau} \quad K = \frac{MC_4}{\sigma^4} - 3
\end{equation}

\subsection{Teste de Concordanta si Independenta}

Statistica testului $\chi^2$ de concordanta: Masoara diferenta dintre frecventele observate ($fa$) si cele asteptate ($fe$).
\begin{equation}
\chi_{\text{Calculat}}^2 = \sum_{i=1}^{m} \frac{(fa_i - fe_i)^2}{fe_i}
\end{equation}

Statistica testului Smirnov-Kolmogorov: Diferenta maxima absoluta dintre functia de repartitie empirica ($Fe$) si cea teoretica ($F$).
\begin{equation}
D = \max_{j} |Fe(x_{(j)}) - F(x_{(j)})|
\end{equation}

Statistica testului $\chi^2$ de independenta (frecvente absolute): Testeaza daca exista o asociere între doua variabile calitative.
\begin{equation}
\chi_{\text{Calculat}}^2 = \sum_{i=1}^{p} \sum_{j=1}^{q} \frac{(n_{ij} - ne_{ij})^2}{ne_{ij}}
\end{equation}

Statistica $\chi^2$ de independenta (frecvente relative): Varianta de calcul bazata pe proportii (frecvente *f*).
\begin{equation}
\chi_{\text{Calculat}}^2 = T \cdot \sum_{i=1}^{p} \sum_{j=1}^{q} \frac{(f_{ij} - f_{i\bullet} f_{\bullet j})^2}{f_{i\bullet} f_{\bullet j}}
\end{equation}

\subsection{Relatia dintre Variabile Cantitative}

Modelul de regresie liniara simpla: Relatia dintre *y* si *x*, incluzând un termen de eroare *e*.
\begin{equation}
y_i = ax_i + b + e_i
\end{equation}

Covarianta: Masoara tendinta a doua variabile de a se modifica împreuna.
\begin{equation}
\text{Cov}(X,Y) = \frac{1}{n} \sum_{i=1}^{n} (x_i - \overline{x})(y_i - \overline{y})
\end{equation}

Coeficientii regresiei liniare: *a* este panta (calculata cu covarianta) si *b* este interceptul.
\begin{equation}
a = \frac{\text{Cov}(X,Y)}{\text{Var}(X)} \quad \text{si} \quad b = \overline{y} - a\overline{x}
\end{equation}

Descompunerea variantei: Varianta totala = Varianta explicata de model + Varianta reziduala.
\begin{equation}
\text{Var}(y) = \text{Var}(ax+b) + \text{Var}(e)
\end{equation}

Coeficientul de determinare ($R^2$): Proportia din varianta lui *y* care este explicata de *x*.
\begin{equation}
R^2(x,y) = \frac{\text{Var}(ax+b)}{\text{Var}(y)} = \frac{\text{Cov}(x,y)^2}{\text{Var}(x)\text{Var}(y)}
\end{equation}

Coeficientul de corelatie liniara (R): Masoara puterea si directia legaturii liniare (de la -1 la 1).
\begin{equation}
R = \frac{\text{Cov}(x,y)}{\sigma_x \sigma_y}
\end{equation}

\section{Document 2: Analiza în Componente Principale (Spatiul Instantelor)}

Definitia componentei principale (combinatie liniara): Componenta *k* este o suma ponderata (combinatie liniara) a variabilelor originale *X*.
\begin{equation}
C_k = a_{1k}X_1 + a_{2k}X_2 + \cdots + a_{mk}X_m
\end{equation}

Solutia (Problema valorilor si vectorilor proprii): Solutia $a_1$ este vectorul propriu al matricei de covarianta $\frac{1}{n} X^t X$.
\begin{equation}
\frac{1}{n} X^t X a_1 = \lambda a_1
\end{equation}

Varianta explicata este egala cu valoarea proprie: Varianta maxima explicata de componenta este $\lambda$, valoarea proprie corespunzatoare vectorului $a_1$.
\begin{equation}
\frac{1}{n} a_1^t X^t X a_1 = \lambda
\end{equation}

\section{Document 3: Analiza în Componente Principale (Evaluare)}

\subsection{Deducere (Spatiul Variabilelor)}

Varianta explicata de axa k: Varianta (inertia) componentei *k* este egala cu valoarea proprie $\alpha_k$.
\begin{equation}
\text{Varianta}(C_k) = \alpha_k
\end{equation}

Suma $R^2$ este egala cu valoarea proprie: Varianta explicata ($\alpha_k$) este, de asemenea, suma corelatiilor la patrat dintre componenta *k* si toate variabilele originale.
\begin{equation}
\sum_{j=1}^{m} R^2(C_k, X_j) = \alpha_k
\end{equation}

Contributia instantei *i* la varianta axei *j*: Arata ce procent din varianta totala a axei *j* este 'generat' de punctul *i*.
\begin{equation}
\beta_{ij} = \frac{1}{n} \cdot \frac{c_{ij}^2}{\alpha_j}
\end{equation}

Comunalitatea variabilei *Xj* (pentru primele *s* componente): Cât la suta din varianta variabilei originale *Xj* este 'capturata' de primele *s* componente.
\begin{equation}
\text{Comunalitate}(X_j) = \sum_{k=1}^{s} R(X_j, C_k)^2
\end{equation}

Corelatii factoriale (Factor Loadings) - vectorul de corelatii: Calculeaza corelatia dintre componenta *k* si *toate* variabilele originale. Aceasta este formula folosita în problema din imagine.
\begin{equation}
R_k = a_k \sqrt{\alpha_k}
\end{equation}


\end{document}